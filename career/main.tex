\documentclass{article}
\usepackage[UTF8]{ctex}
\usepackage{geometry}
\usepackage{multirow}
\usepackage{natbib}
\usepackage{indentfirst} 
\geometry{left=3.18cm,right=3.18cm,top=2.54cm,bottom=2.54cm}
\usepackage{graphicx}
\pagestyle{plain}	
\usepackage{setspace}
\usepackage{enumerate}
\usepackage{caption2}
\usepackage{datetime} %日期
\renewcommand{\today}{\number\year 年 \number\month 月 \number\day 日}
\renewcommand{\captionlabelfont}{\small}
\renewcommand{\captionfont}{\small}
\begin{document}

\begin{figure}
    \centering
    \includegraphics[width=8cm]{upc.png}

    \label{figupc}
\end{figure}

	\begin{center}
		\quad \\
		\quad \\
		\heiti \fontsize{45}{17} \quad \quad \quad 
		\vskip 1.5cm
		\heiti \zihao{2} 《计算科学导论》个人职业规划
	\end{center}
	\vskip 2.0cm
		
	\begin{quotation}
% 	\begin{center}
		\doublespacing
		
        \zihao{4}\par\setlength\parindent{7em}
		\quad 

		学生姓名:\underline{\qquad  刘国威 \qquad \qquad}

		学\hspace{0.61cm} 号:\underline{\qquad 1604010109\qquad}
		
		专业班级:\underline{\qquad 计科1703 \qquad  }
		
        学\hspace{0.61cm} 院:\underline{计算机科学与技术学院}
% 	\end{center}
		\vskip 1.5cm
		\centering
		\begin{table}[h]
            \centering 
            \zihao{4}
            \begin{tabular}{|c|c|c|c|c|c|c|c|c|}
            % 这里的rl 与表格对应可以看到,姓名是r,右对齐的;学号是l,左对齐的;若想居中,使用c关键字。
                \hline
                \multicolumn{5}{|c|}{分项评价} &\multicolumn{2}{c|}{整体评价}  & 总    分 & 评 阅 教 师\\
                \hline
                自我 & 环境 & 职业 & 实施 & 评估与 & 完整性 & 可行性 &\multirow{2}*{} &\multirow{2}*{}\\
                分析& 分析& 定位 & 方案 & 调整 & 20\% & 20\% & ~&~ \\\            
                10\% & 10\% & 15\% & 15\% & 10\% & &  &~ &~\\
                \cline{1-7} 
                & & & & & & & ~&~ \\
                & & & & & & & ~&~ \\
                \hline      
            \end{tabular}
        \end{table}
		\vskip 2cm
		\today
	\end{quotation}

\thispagestyle{empty}
\newpage
\setcounter{page}{1}
% 在这之前是封面,在这之后是正文
\section{自我分析}
\subsection{自然条件}
% table
\begin{table}[h]
	\centering 
	\zihao{4}
	\begin{tabular}{|c|c|c|c|c|}
		\hline
		\multicolumn{5}{|c|}{自然条件分析} \\
		\hline
		性别 & 年龄 & 身体条件 & 健康状况 & 居住城市 \\
		\cline{1-5} 
		男& 21 &健壮 & 健康& 山东青岛 \\
		\hline      
	\end{tabular}
\end{table}

\par
\subsection{性格分析}
目前,心理学家们普遍认为, 在通常的情况下, 人的性格类型可分为胆汁质,
多血质,粘液质和抑郁质四种,其性格特点分别是:\par
第一,胆汁质型(反应速度快,意志力强):性情急噪,神经系统坚强,不
怕困难,缺乏自制力,缺乏持久而有系统的进行工作能力。\par
第二,抑郁质型(反应速度慢,意志力弱):多愁善感,神经系统较敏感,
仰制性较强,固执而容易生气,不善于交际,不能禁受长期的紧张工作。\par
第三,多血质型(反应速度快,意志力弱):见异思迁,神经系统坚强,感
觉和行动都是均衡的,活泼好动,善于交际,能适应各种情况,常常容易作出妥
协。\par
第四,粘液质型(反应速度快,意志力弱):性情孤僻,感觉和行动是均衡
的,表情不显于色,感情稳定,反应迟钝,难适应生活条件的改变,工作埋头苦
干。\par
根据上述性格特点分析,
我认为我的性格更像是抑郁质型和粘液质型的混合体。
我性格有些内向,不太喜欢表现自己的感受。适应能力较差,喜欢固定的生活方式。
情感体验不深但有些敏感。
我能够意识到自己和他人的情感,但有时会忽视他们。
性格有些固执,甚至有点偏执。认定的事情,很难发生改变。
\par
就像一枚硬币有正反面,一件事并不总是好或者总是坏。
我遇事更喜欢刨根问底。这个学期学习网络原理的时候,我对ICMP的差错报告报文感到好奇,虽然书上已经给出了详细的报文结构,但我更想亲自用wireshake抓包
分析一下。但是ICMP的差错报告报文是由路由器提供给源主机的,在宿舍里有的只是一台交换机。于是,我先尝试在virtualBox的虚拟机下进行抓包实验,可想而知,虚拟机和宿主机公用一块网卡,所以实验失败。但是,我没有放弃。我又尝试将宿舍里的一台台式机跟自己的笔记本用网线连接在一起,配置好静态IP地址。开启台式机的路由转发功能,从笔记本经台式机向一个不存在的IP地址发送请求。最终,终于捕获了一个ICMP的差错报告报文。
\par
\newpage
\subsection{教育与学习经历}
无论是小学还是中学,我都是在我们那边的镇上读的。
初中毕业之后,我进入了黄岛区第二中学就读。
高中毕业之后,我又顺利考入中国石油大学(华东)就读。
从小到大,我的成绩一直都十分的稳定,处在一个中等的水平。
\par
2016年的时候,我进入石大的机电院,就读的是机械设计制造与自动化专业。在机自的日子,生活像是一杯白开水,平淡无奇。
直到大一下的时候,我在图书馆里看到了Mark Allen Weiss的《数据结构与算法分析》。书中有关各种数据结构的插图吸引了我。怀着好奇的心理,我学完了书的前四章,并在机房的电脑上实现了它们。从那时开始,我对计算机科学技术这个专业充满了向往。\par
大二时,我经过慎重的思考,从机自转到了计科。\par
\par
\subsection{工作与社会阅历}
直到现在,我的工作重心一直都是学习。学习一些有趣的知识,学一些不错的技术。
在大一和大二的时候,我曾经短暂的担任过一段时间的家教。辅导过一名初三学生的数学,还有一名高三艺考生的数学。
但是,时间都比较短。
除此之外,我的工作和社会阅历,都只停留在过年过节时,随父母走街串巷,拜访亲戚了。
\par
\subsection{知识、技能与经验}
因为学过两个专业的缘故,我既了解一些机械行业相关的知识,也会一些计算机方面的技术。
我工程制图的能力还不错,2018年的时候,还拿过机电杯省赛的一等奖。除此之外,我也参加过一些软件相关的比赛,我也拿过2018年齐鲁软件大赛的一等奖。

\par
\subsection{兴趣爱好与特长}
我喜欢书法、象棋、长跑、计算机等。我曾经是机电院长跑队的一员,所以十分喜爱长跑。计算机方面,我最喜欢偏硬件方面的学科,比如汇编、组成原理等。此外,我对linux内核十分的感兴趣,最想做的事情之一便是写一个toy操作系统内核。
\par
\newpage
\section{环境分析}
\subsection{社会环境分析}
中国现在正处于近两百年以来最好的历史时期。虽然社会上还有许多的体制弊端,还有许多没有解决的矛盾,但是政治上比较稳定,法制化进程已经开始,市场经济已经初步形成并步入正轨。二十一世纪的中华大地充满各种人才成长发展的机遇。\par
但是我们也要看到,人才的竞争日趋激烈,大学生就业难、失业率居高不下等等,都使我们的就业环境看起来不容乐观,这就更需要在分析好社会现状的基础下,有针对性地做好自己的职业生涯规划。\par
总体来说,我们现在面临一个非常好的宏观环境,社会安定,政治稳定,经济发展迅速,并与全球一体化接轨,法制建设不断完善,文化繁荣自由,尖端技术、高新技术突飞猛进。因此,在这个大前提之下,我们需要特别注意的是职业环境的变化。\par
\subsection{家庭环境分析}
我来自黄岛区下面一个偏僻的小乡镇。家中有四名成员,家庭小康。\par
父母的对我的要求比较宽松,很理解我,在各方面都会充分尊重我的意见,通常状况都不干涉我的决定。姐姐年长我不少,十分地痛爱我。\par
父母都希望我可以选择一个相对稳定的工作,希望我可以成为一名公务员。\par
\subsection{职业环境分析}
IT行业处在结构性升级的过程中以适应新的行业发展趋势。在结构性调整的过程中,通常伴随着人才结构的调整,一部分技术人员面临被裁员的风险,而同时还会招聘一批新技术人才。\par
从市场总体发展情况来看,伴随着中国逐渐成为全球IT企业关注的重心,中国IT业市场的竞争日趋激烈。中国不但已经成为全球重要的IT制造中心,同
时也逐渐步入全球IT研发中心的角色。良好的国内经济环境使中国IT行业市场继续保持平稳的增长, 蓬勃发展的中国软件与IT服务市场吸引了众多新进者。
展望未来,中国国民经济将继续保持稳步的增长,中国行业和企业信息化建设进一步深化,消费者消费水平逐步提高,中国IT行业市场具有良好的发展前景。\par


\subsection{地域与人际环境分析}
想要留在青岛工作。\par
青岛地处北温带季风区域,属温带季风气候。市区由于海洋环境的直接调节,受来自洋面上的东南季风及海流、水团的影响,故又具有显著的海洋性气候特点。空气湿润,雨量充沛,温度适中,四季分明。\par
文化受儒家文化的影响比较大。\par
青岛IT行业的发展情况比不得北上广深这样的一线大城市。但是青岛的IT行业的发展潜力是非常巨大的。我相信未来青岛的IT行业市场需求将不断增大、提升速度将不断加快、就业范围将变得更广。\par
没有什么人脉,但是人际关系良好。\par



\section{职业定位}
\subsection{行业领域定位与理由}
IT行业无疑发展是十分迅速的。IT行业的发展,也带动着其他行业的发展。
\par
\subsection{职业岗位起点定位与理由}
想选择地职业岗位是硬件开发师。\par
软件工程师的待遇在初期是明显的高于硬件工程师并且软件工程师地成长的方向非常的多。虽然硬件工程师也是有很多成长方向,但是成长周期就不如软件工程师。\par
但是,硬件开发容易成熟。就像计算机主板,功能强大,一旦做出来后,就有可能是万能的。输入输出接上不同的设备,CPU再刷上不同的程序,那就是不同的系统,表现出不同的功能。现在,随着物联网地发展,对于硬件开发也越来越重视了。\par
我相信硬件工程师在未来一定会崛起,受到欢迎的。因为中国需要一颗中国芯,中国也需要智能制造。
中兴地事件告诫我们,只有拥有自己地中国芯,中国才能在国际上能够立足,不被西方强国所压迫。\par

\subsection{职业目标与可行性分析}
\begin{enumerate}[(1)]
	\item 短期目标(大学4年)\par
	大学期间打好基础,掌握所需要的必备技能。本科毕业之后,想去山东大学进一步读书深造,从事硬件设计开发方向。或者参加公务员考试。
	\item 中长期目标(5-10年)。\par
	如果继续读书深造,便潜心学习。在学习的同时不断地提升自己的实践能力。毕业之后,进入一家硬件相关的公司,从事硬件开发设计岗。\par
	期待的薪资是8000 至 10K。
\end{enumerate}
\newpage
\section{实施方案}
\begin{enumerate}[1、]
	\item 目前处在大三。现在最重要的事情就是努力学习,提高GPA,争取学分绩上91.5。希望明年可以争取到保研的机会。如果可以争取到这个机会,我想在硬件开发/嵌入式这个方向进一步深造。
	\item 如果有机会读研究生。在研究生期间,我会继续潜心学习,认真钻研,努力学得一技之长。在研究生的后半期,开始多走向社会,多历练,提高自己的实践能力。
	\item 研究生毕业之后,争取进入一家硬件开发公司,从事硬件开发和设计。在不断提高自己技术的同时,成为这家公司的技术骨干。
	\item 如果明年的保研不顺利的话,开始全力备考公务员。争取可以通过山东省的省考,成为一名公务员。然后在自己的岗位上发光发热,为社会服务。
\end{enumerate}
\par 



\section{评估与调整}
\subsection{评估时间}
2020年12月 、2021年7月 、2024年7月 、2027年12月 、2030年12月
\par
\subsection{评估内容}
\begin{enumerate}[1、]
	\item 2020年12月,是否去山东大学进一步深造
	\item 2021年7月,是否通过省考成为一名公务员
	\item 2024年7月,是否具有了不错的技术并在公司中成为中流砥柱,或者在底层扎根为人民全心全意服务
	\item 2027年12月,经济情况是否稳定,是否已经有房有车
	\item 2030年12月,是否结婚,有没有孩子,是否幸福美满
\end{enumerate}



\par
\subsection{调整原则}
由于影响职业生涯规划的因素很多,且大都处于动态变化之中,因此职业生涯规划应定期评估,并根据影响因素的变化和实施结果的情况及时作出调整,这样才能保证其行之有效。\par
首先,调整时,要根据具体的情况,实事求是的分析,不能臆断。\par
其次,百善孝为先。父母日益年迈,工作地点应尽可能选在父母身边。\par
最后,工作不应过分地挤占生活的时间。工作不是生命的唯一,我们还有许多重要的事去做、许多重要的人去陪。\par
\par




\end{document}
